\documentclass[dvipdfmx,a4paper,11pt]{jsarticle}

% 数式
\usepackage{amsmath,amsfonts,amssymb,mathtools,bm,titlesec,amsthm,ulem,siunitx,braket,url,physics,otf,bussproofs,enumitem,stmaryrd}
% 画像
\usepackage[dvipdfmx]{graphicx}
\usepackage{tikz}

\mathtoolsset{showonlyrefs} % 参照した式のみ番号を表示
% \everymath{\displaystyle}

\renewcommand{\thesubsection}{\arabic{subsection}}
\usepackage{titlesec}
\titleformat{\subsection}[hang]{\normalfont\normalsize\bfseries}{\S\thesubsection}{1em}{}
\renewcommand{\theenumi}{\arabic{subsection}.\arabic{enumi}}
\renewcommand{\labelenumi}{\textbf{\theenumi}}


\begin{document}

\title{Tu多様体 回答}
\author{tRue}
\date{\today}
\maketitle

\section*{節末問題}

\subsection{ユークリッド空間上の滑らかな関数}
\begin{enumerate}
  \item \textgt{$C^2$級だが$C^3$級でない関数}\\
  $f\colon\mathbb{R}\to\mathbb{R}$を$f(x)=x^{1/3}$とし,
  $g\colon\mathbb{R}\to\mathbb{R}$を
  \begin{align}
    g(x)=\int_0^x f(t)dt=\int_0^xt^{1/3}dt=\frac{3}{4}x^{4/3}
  \end{align}
  とする.関数$h(x)=\int_0^xg(t)dt$は$C^2$級
  だが$x=0$で$C^3$級ではないことを示せ.
  \begin{proof}
    $h(x)$の1階微分,2階微分,3階微分を計算する.
    \begin{align}
      h'(x)&=g(x)=\frac{3}{4}x^{4/3},\\
      h''(x)&=g'(x)=f(x)=x^{1/3},\\
      h'''(x)&=f'(x)=\frac{1}{3}x^{-2/3}.
    \end{align}
    したがって,$h''(x)$は全ての$x$で定義されるが,
    $h'''(x)$は$x=0$で定義されない.
    よって,$h(x)$は$C^2$級だが$x=0$で$C^3$級ではない.
  \end{proof}
\end{enumerate}

\setcounter{subsection}{7}
\subsection{接空間}
\begin{enumerate}
  \setcounter{enumi}{9}
  \item \textgt{極大値}\\
  \label{secprob:8.10}
  多様体上の実数値関数$f\colon M\to\mathbb{R}$が
  $p\in M$において極大値をもつとは,
  $f(p)\geq f(q)$がすべての$q\in U$について成り立つような
  $p$の近傍$U$が存在することである.
  \begin{enumerate}
    \item 開区間$I$上で定義されている微分可能な関数
    $f\colon I\to\mathbb{R}$が$p\in I$において極大値をもつ
    ならば,$f'(p)=0$であることを示せ.
    \begin{proof}
      $f(p)\geq f(q)$がすべての$q\in I$について成り立つように
      $I$として取り直しても良い.\\
      $q<p$で$f(q)$が増加,$p<q$で$f(q)$が減少することを
      踏まえると以下2つが成り立つ.
      \begin{align}
        \lim_{q\to p^-}\frac{f(q)-f(p)}{q-p}&\leq 0\\
        \lim_{q\to p^+}\frac{f(q)-f(p)}{q-p}&\geq 0
      \end{align}
      $f$が微分可能であるため,左極限と右極限は
      等しくなければならず,$f'(p)=0$が従う.
    \end{proof}
    \item $C^\infty$級関数$f\colon M\to\mathbb{R}$が
    極大値をとる点は$f$の臨界点であることを証明せよ.
    \begin{proof}
      $p\in M$を$f$が極大値をとる点とし,
      $X_p\in T_p M$を接ベクトルとする.
      $c(t)$を始点$p$における速度ベクトルが$X_p$であるような
      $M$上の曲線とすると,
      $f\circ c\colon \mathbb{R}\to\mathbb{R}$
      は$0$において極大値をもち,(a)より
      \begin{align}
        &(f\circ c)'(0)=0\\
        \implies&
        (f\circ c)_*\qty(\frac{d}{dt})
        =f_{*,p}(X_p)=0
      \end{align}
      となる.これは任意の$X_p\in T_p M$について成り立つため,
      $f_{*,p}$は零写像であり,$p$は$f$の臨界点である.
    \end{proof}
  \end{enumerate}
\end{enumerate}

\setcounter{subsection}{10}
\subsection{滑らかな写像の階数}
\begin{enumerate}
  \item \textgt{球面の接ベクトル}\\
  $\mathbb{R}^{n+1}$における単位球面$S^n$は,
  方程式$\sum_{i=1}^{n+1}(x^i)^2=1$によって定義される.
  $p=(p^1,\dots,p^{n+1})\in S^n$に対して,
  \begin{align}
    X_p=\left.\sum a^i\partial/\partial x^i
    \right|_{p}\in T_p\mathbb{R}^{n+1}
  \end{align}
  が点$p$で$S^n$に接するための必要十分条件は,
  $\sum a^i p^i=0$であることを示せ.
  \begin{proof}
    $f\colon\mathbb{R}^{n+1}\to\mathbb{R}$を
    \begin{align}
      f(x^1,\dots,x^{n+1})=\sum_{i=1}^{n+1}(x^i)^2-1
    \end{align}
    とすると,$S^n=f^{-1}(0)$である.\\
    $X_p$が点$p$で$S^n$に接するなら,ある
    曲線$c\colon\mathbb{R}\to S^n$が存在して,
    $c(0)=p$, $c'(0)=X_p$を満たす(命題8.16)
    \footnote{
      ここは必要十分条件だと思うが,
      Tu多様体の回答では必要性のみを用いていたのでそれに沿った.
      一般には十分ではないのかもしれないが,私はよくわかっていない.
    }
    .
    $i\colon S^n\to\mathbb{R}^{n+1}$を包含写像とすると,
    このような曲線$c(t)$について,
    ここで,$f\circ i\circ c\colon\mathbb{R}\to\mathbb{R}$は
    全ての$t$で0となるため,
    \begin{align}
      0&=\frac{d}{dt}(f\circ i\circ c)(t)\\
      &=(f\circ i\circ c)_*\qty(\frac{d}{dt}\bigg|_{t})\\
      &=f_{*,c(t)}\qty(c'(t))\\
      &=\sum_{i=1}^{n+1}
      \pdv{f}{x^i}\bigg|_{c(t)}\dot{c}^i(t)
    \end{align}
    となる\footnote{
      本来は終域の接空間の基底
      $\frac{\partial}{\partial z}\bigg|_0$
      を掛ける必要があるが,ここでは省略している.
    }.$t=0$のとき,
    \begin{align}
      \sum_{i=1}^{n+1}
      \pdv{f}{x^i}\qty(p)\cdot a^i&=0\\
      \iff
      \sum_{i=1}^{n+1}2p^i\cdot a^i&=0\\
      \iff
      \sum_{i=1}^{n+1}a^ip^i&=0
    \end{align}
    となる.\\
    $T_p S^n$と$\sum_{i=1}^{n+1}a^ip^i=0$を満たす
    $T_p\mathbb{R}^{n+1}$の部分集合はどちらも同じ次元をもつ
    ベクトル空間で,上述の議論から前者は後者に含まれるため,同型である.
  \end{proof}
  \item\textgt{平面曲線の接ベクトル}\\
  \begin{enumerate}
    \item $i\colon S^1\hookrightarrow\mathbb{R}^2$を
    単位円周の包含写像とする.
    この問題では,$x,y$を$\mathbb{R}^2$の標準座標とし,
    $\overline{x},\overline{y}$をその$S^1$への制限とする.
    よって,$\overline{x}=i^*x,\overline{y}=i^*y$である.
    上半円周$U=\{(a,b)\in S^1\mid b>0\}$
    においては,$\overline{x}$は局所座標であり,
    ゆえに$\partial/\partial \overline{x}$
    が定義されている.
    $p\in U$に対して\begin{align}
      i_*\qty(\pdv{}{\overline{x}}\bigg|_{p})
      =\qty(\pdv{}{x}+\pdv{\overline{y}}{\overline{x}}
      \pdv{}{y})\bigg|_p
    \end{align}
    を証明せよ.したがって,
    $i\colon T_p S^1\to T_p\mathbb{R}^2$は単射であるが,
    $\partial/\partial \overline{x}|_p$は
    $\partial/\partial x|_p$と同一視することはできない.
    \begin{proof}
      \begin{align}
        i_*\qty(\pdv{}{\overline{x}}\bigg|_{p})
        &=\qty(\pdv{x\circ i}{\overline{x}}\pdv{}{x}
        +\pdv{y\circ i}{\overline{x}}\pdv{}{y})\bigg|_p\\
        &=\qty(\pdv{\overline{x}}{\overline{x}}\pdv{}{x}
        +\pdv{\overline{y}}{\overline{x}}\pdv{}{y})
        \bigg|_p\\
        &=\qty(\pdv{}{x}
        +\pdv{\overline{y}}{\overline{x}}\pdv{}{y})
        \bigg|_p
      \end{align}
    \end{proof}
    \item $\mathbb{R}^2$における滑らかな曲線$C$について,
    $x$の$C$への制限である$\overline{x}$が局所座標になるような
    $C$のチャート$U$をとり,(a)の結果を一般化せよ.
    \begin{proof}
      (a)の変形は曲線の方程式に依らない.
      適切にチャートが取れていれば結果は同様.
    \end{proof}
  \end{enumerate}
  \item \textgt{コンパクトな多様体上の滑らかな写像の臨界点}\\
  コンパクトな多様体$N$から$\mathbb{R}^m$への滑らかな写像
  $f$は臨界点をもつことを示せ.
  \begin{proof}
    第1成分への射影$\pi\colon\mathbb{R}^m\to\mathbb{R}$を用いて,
    $\pi\circ f\colon N\to\mathbb{R}$を考える.
    $N$はコンパクトで,$f$と$\pi$は連続であるから
    $\pi\circ f(N)$はコンパクトで,そのため有界閉集合.
    よって,$\pi\circ f$は最大値(と最小値)をもつ.\\
    $p\in N$を$\pi\circ f$の最大値を与える点とし,
    $(U,x^1,\dots,x^n)$を$p$を含むチャートとする.
    以下を考える.
    \begin{align}
      (\pi\circ f)_*\qty(\pdv{}{x^i}\bigg|_{p})
      &=\pdv{f^1}{x^i}\qty(p)\cdot \pdv{}{f^1}\bigg|_{f(p)}\\
      &=0
    \end{align}
    したがって,$f_*\qty(\partial/\partial x^i|_{p})$の
    第1成分は全ての$i$について0であり,
    そのため$f_*$は$p$で全射でない.つまり,$p$は臨界点である.
  \end{proof}
  \emph{別解1}\footnote{
    Tu多様体の回答
  }
  \begin{proof}
    $f$が臨界点をもたないと仮定すると,$f$は沈め込み
    (つまり,適切なチャートを取れば$f$は射影).
    ここで$\pi\colon\mathbb{R}^m\to\mathbb{R}$を
    第1成分への射影とすると,
    $\pi\circ f\colon N\to\mathbb{R}$も沈め込みである
    (つまり,適切なチャートを取れば$\pi\circ f$は$x^1$).
    しかし,$\pi\circ f(N)$はコンパクトであるから最大値を持ち,
    その点で臨界点となってしまう\footnote{
      \ref{secprob:8.10}参照.
    }ため矛盾.
  \end{proof}
  \emph{別解2}\footnote{
    Tu多様体の別解
  }
  \begin{proof}
    $f$が臨界点をもたないと仮定すると,$f$は沈めこみ.
    系11.6より$f$は開写像,滑らかという仮定も踏まえて同相写像となる.
    しかし,$N$はコンパクトであるから$f(N)$は開なコンパクト集合
    となるが,$\mathbb{R}^m$の部分集合でその条件を満たすものは
    空集合のみで,矛盾する.
  \end{proof}
  \item \textgt{包含写像の微分}\\
  単位球面$S^2$の上半球面においては,
  \begin{align}
    u(a,b,c)=a\text{ および }v(a,b,c)=b
  \end{align}
  で与えられる座標写像$\phi=(u,v)$がある.
  ゆえに,半球面上の任意の点$p=(a,b,c)$において,
  偏導関数$\partial/\partial u|_p,\partial/\partial v|_p$
  は$S^2$の接ベクトルである.
  $i\colon S^2\hookrightarrow\mathbb{R}^3$を包含写像とし,
  $x,y,z$を$\mathbb{R}^3$の標準座標とする.
  微分$i_*\colon T_p S^2\to T_p\mathbb{R}^3$は
  $\partial/\partial u|_p,\partial/\partial v|_p$を
  $T_p\mathbb{R}^3$に写す.したがって,
  定数$\alpha^i,\beta^i,\gamma^i$を用いて
  \begin{align}
    i_*\qty(\pdv{}{u}\bigg|_{p})
    &=\alpha^1\pdv{}{x}\bigg|_{p}
    +\beta^1\pdv{}{y}\bigg|_{p}
    +\gamma^1\pdv{}{z}\bigg|_{p},\\
    i_*\qty(\pdv{}{v}\bigg|_{p})
    &=\alpha^2\pdv{}{x}\bigg|_{p}
    +\beta^2\pdv{}{y}\bigg|_{p}
    +\gamma^2\pdv{}{z}\bigg|_{p}
  \end{align}
  と書ける.$i=1,2$について,$(\alpha^i,\beta^i,\gamma^i)$を求めよ.\\
  \begin{align}
    i_*\qty(\pdv{}{u}\bigg|_{p})
    &=\qty(\pdv{x\circ \phi^{-1}}{u}\pdv{}{x}
    +\pdv{y\circ \phi^{-1}}{u}\pdv{}{y}
    +\pdv{z\circ \phi^{-1}}{u}\pdv{}{z})\bigg|_{p}\\
    &=\qty(\pdv{u}{u}\pdv{}{x}
    +\pdv{v}{u}\pdv{}{y}
    +\pdv{\sqrt{1-u^2-v^2}}{u}\pdv{}{z})\bigg|_{p}\\
    &=\qty(1\cdot\pdv{}{x}
    +0\cdot\pdv{}{y}
    -\frac{u}{\sqrt{1-u^2-v^2}}\cdot\pdv{}{z})\bigg|_{p}
  \end{align}
  同様に
  \begin{align}
    i_*\qty(\pdv{}{v}\bigg|_{p})
    &=\qty(0\cdot\pdv{}{x}
    +1\cdot\pdv{}{y}
    -\frac{v}{\sqrt{1-u^2-v^2}}\cdot\pdv{}{z})\bigg|_{p}
  \end{align}
  である.よって,
  \begin{align}
    (\alpha^1,\beta^1,\gamma^1)&=\left(1,0,-\frac{u}{\sqrt{1-u^2-v^2}}\right),\\
    (\alpha^2,\beta^2,\gamma^2)&=\left(0,1,-\frac{v}{\sqrt{1-u^2-v^2}}\right).
  \end{align}
  \item \textgt{コンパクトな多様体の1対1のはめ込み}\\
  $N$がコンパクトな多様体のとき,1対1のはめ込み$f\colon N\to M$
  は埋め込みであることを証明せよ.
  \begin{proof}
    $f(N)$に部分空間位相を入れたものが$f$の下で$N$が同相であることを
    示せば良い.連続であることは$V\in \mathcal{O}_{f(N)}$に対して
    $f(N)\cap V'=V$なる$V'\in\mathcal{O}_M$が存在し,
    \begin{align}
      f^{-1}(V)=f^{-1}(f(N)\cap V')=f^{-1}(V')\in\mathcal{O}_N
    \end{align}
    より従う.\\
    開写像であることを示す.
    多様体の定義にハウスドルフ性が含まれていたことを踏まえると,
    $f$はコンパクト空間からハウスドルフ空間への連続写像であるから
    ,閉写像である\footnote{
      コンパクト空間の閉部分集合はコンパクトで,
      コンパクト空間の連続写像による像はコンパクトとなり,
      ハウスドルフ空間のコンパクト部分集合は閉集合であるため.
    }.
    さて,任意に$U\in\mathcal{O}_N$を取ると,以下が成り立つ.
    \begin{align}
      U\in\mathcal{O}_N&\implies N\setminus U\in\mathcal{C}_N\\
      &\implies f(N\setminus U)\in\mathcal{C}_M\\
      &\implies M\setminus f(N\setminus U)\in\mathcal{O}_M
    \end{align}
    ここで,$M\setminus f(N\setminus U)$を$V$とおくと,
    $f(U)=f(N)\cap V$である.
    よって任意の$U$に対して$f(U)=f(N)\cap V$なる$V\in\mathcal{O}_M$が
    存在し,
    これはすなわち$f(U)$が$f(N)$の部分空間位相に関して開であることを
    意味する.\\
    以上より$f\colon N\to f(N)$は$f(N)$の部分空間位相
    に関して同相写像であり,したがって$f$は埋め込みである.
  \end{proof}
\end{enumerate}


\end{document}