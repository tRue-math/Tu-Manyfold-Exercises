\documentclass[dvipdfmx,a4paper,11pt]{jsarticle}

% 数式
\usepackage{amsmath,amsfonts,amssymb,mathtools,bm,titlesec,amsthm,ulem,siunitx,braket,url,physics,otf,bussproofs,enumitem,stmaryrd}
% 画像
\usepackage[dvipdfmx]{graphicx}
\usepackage{tikz}

\mathtoolsset{showonlyrefs} % 参照した式のみ番号を表示
% \everymath{\displaystyle}

\renewcommand{\thesubsection}{\arabic{subsection}}
\usepackage{titlesec}
\titleformat{\subsection}[hang]{\normalfont\normalsize\bfseries}{\S\thesubsection}{1em}{}
\renewcommand{\theenumi}{\arabic{subsection}.\arabic{enumi}}
\renewcommand{\labelenumi}{\textbf{\theenumi}}


\begin{document}

\title{Tu多様体 回答}
\author{tRue}
\date{\today}
\maketitle

\section*{節末問題}

\subsection{ユークリッド空間上の滑らかな関数}
\begin{enumerate}
  \item \textgt{$C^2$級だが$C^3$級でない関数}\\
  $f\colon\mathbb{R}\to\mathbb{R}$を$f(x)=x^{1/3}$とし,
  $g\colon\mathbb{R}\to\mathbb{R}$を
  \begin{align}
    g(x)=\int_0^x f(t)dt=\int_0^xt^{1/3}dt=\frac{3}{4}x^{4/3}
  \end{align}
  とする.関数$h(x)=\int_0^xg(t)dt$は$C^2$級
  だが$x=0$で$C^3$級ではないことを示せ.
  \begin{proof}
    $h(x)$の1階微分,2階微分,3階微分を計算する.
    \begin{align}
      h'(x)&=g(x)=\frac{3}{4}x^{4/3},\\
      h''(x)&=g'(x)=f(x)=x^{1/3},\\
      h'''(x)&=f'(x)=\frac{1}{3}x^{-2/3}.
    \end{align}
    したがって,$h''(x)$は全ての$x$で定義されるが,
    $h'''(x)$は$x=0$で定義されない.
    よって,$h(x)$は$C^2$級だが$x=0$で$C^3$級ではない.
  \end{proof}
\end{enumerate}

\setcounter{subsection}{7}
\subsection{接空間}
\begin{enumerate}
  \setcounter{enumi}{9}
  \item \textgt{極大値}\\
  \label{secprob:8.10}
  多様体上の実数値関数$f\colon M\to\mathbb{R}$が
  $p\in M$において極大値をもつとは,
  $f(p)\geq f(q)$がすべての$q\in U$について成り立つような
  $p$の近傍$U$が存在することである.
  \begin{enumerate}
    \item 開区間$I$上で定義されている微分可能な関数
    $f\colon I\to\mathbb{R}$が$p\in I$において極大値をもつ
    ならば,$f'(p)=0$であることを示せ.
    \begin{proof}
      $f(p)\geq f(q)$がすべての$q\in I$について成り立つように
      $I$として取り直しても良い.\\
      $q<p$で$f(q)$が増加,$p<q$で$f(q)$が減少することを
      踏まえると以下2つが成り立つ.
      \begin{align}
        \lim_{q\to p^-}\frac{f(q)-f(p)}{q-p}&\leq 0\\
        \lim_{q\to p^+}\frac{f(q)-f(p)}{q-p}&\geq 0
      \end{align}
      $f$が微分可能であるため,左極限と右極限は
      等しくなければならず,$f'(p)=0$が従う.
    \end{proof}
    \item $C^\infty$級関数$f\colon M\to\mathbb{R}$が
    極大値をとる点は$f$の臨界点であることを証明せよ.
    \begin{proof}
      $p\in M$を$f$が極大値をとる点とし,
      $X_p\in T_p M$を接ベクトルとする.
      $c(t)$を始点$p$における速度ベクトルが$X_p$であるような
      $M$上の曲線とすると,
      $f\circ c\colon \mathbb{R}\to\mathbb{R}$
      は$0$において極大値をもち,(a)より
      \begin{align}
        &(f\circ c)'(0)=0\\
        \implies&
        (f\circ c)_*\qty(\frac{d}{dt})
        =f_{*,p}(X_p)=0
      \end{align}
      となる.これは任意の$X_p\in T_p M$について成り立つため,
      $f_{*,p}$は零写像であり,$p$は$f$の臨界点である.
    \end{proof}
  \end{enumerate}
\end{enumerate}

\setcounter{subsection}{10}
\subsection{滑らかな写像の階数}
\begin{enumerate}
  \item \textgt{球面の接ベクトル}\\
  $\mathbb{R}^{n+1}$における単位球面$S^n$は,
  方程式$\sum_{i=1}^{n+1}(x^i)^2=1$によって定義される.
  $p=(p^1,\dots,p^{n+1})\in S^n$に対して,
  \begin{align}
    X_p=\left.\sum a^i\partial/\partial x^i
    \right|_{p}\in T_p\mathbb{R}^{n+1}
  \end{align}
  が点$p$で$S^n$に接するための必要十分条件は,
  $\sum a^i p^i=0$であることを示せ.
  \begin{proof}
    $f\colon\mathbb{R}^{n+1}\to\mathbb{R}$を
    \begin{align}
      f(x^1,\dots,x^{n+1})=\sum_{i=1}^{n+1}(x^i)^2-1
    \end{align}
    とすると,$S^n=f^{-1}(0)$である.\\
    $X_p$が点$p$で$S^n$に接するなら,ある
    曲線$c\colon\mathbb{R}\to S^n$が存在して,
    $c(0)=p$, $c'(0)=X_p$を満たす(命題8.16)
    \footnote{
      ここは必要十分条件だと思うが,
      Tu多様体の回答では必要性のみを用いていたのでそれに沿った.
      一般には十分ではないのかもしれないが,私はよくわかっていない.
    }
    .
    $i\colon S^n\to\mathbb{R}^{n+1}$を包含写像とすると,
    このような曲線$c(t)$について,
    ここで,$f\circ i\circ c\colon\mathbb{R}\to\mathbb{R}$は
    全ての$t$で0となるため,
    \begin{align}
      0&=\frac{d}{dt}(f\circ i\circ c)(t)\\
      &=(f\circ i\circ c)_*\qty(\frac{d}{dt}\bigg|_{t})\\
      &=f_{*,c(t)}\qty(c'(t))\\
      &=\sum_{i=1}^{n+1}
      \pdv{f}{x^i}\bigg|_{c(t)}\dot{c}^i(t)
    \end{align}
    となる\footnote{
      本来は終域の接空間の基底
      $\frac{\partial}{\partial z}\bigg|_0$
      を掛ける必要があるが,ここでは省略している.
    }.$t=0$のとき,
    \begin{align}
      \sum_{i=1}^{n+1}
      \pdv{f}{x^i}\qty(p)\cdot a^i&=0\\
      \iff
      \sum_{i=1}^{n+1}2p^i\cdot a^i&=0\\
      \iff
      \sum_{i=1}^{n+1}a^ip^i&=0
    \end{align}
    となる.\\
    $T_p S^n$と$\sum_{i=1}^{n+1}a^ip^i=0$を満たす
    $T_p\mathbb{R}^{n+1}$の部分集合はどちらも同じ次元をもつ
    ベクトル空間で,上述の議論から前者は後者に含まれるため,同型である.
  \end{proof}
\end{enumerate}


\end{document}