\documentclass[dvipdfmx,a4paper,11pt]{jsarticle}

% 数式
\usepackage{amsmath,amsfonts,amssymb,mathtools,bm,titlesec,amsthm,ulem,siunitx,braket,url,physics,otf,bussproofs,enumitem,stmaryrd}
% 画像
\usepackage[dvipdfmx]{graphicx}
\usepackage{tikz}

\mathtoolsset{showonlyrefs} % 参照した式のみ番号を表示
% \everymath{\displaystyle}

\renewcommand{\thesubsection}{\arabic{subsection}}
\usepackage{titlesec}
\titleformat{\subsection}[hang]{\normalfont\normalsize\bfseries}{\S\thesubsection}{1em}{}
\renewcommand{\theenumi}{\arabic{subsection}.\arabic{enumi}}
\renewcommand{\labelenumi}{\textbf{\theenumi}}


\begin{document}

\title{Tu多様体 回答}
\author{tRue}
\date{\today}
\maketitle

\section*{節末問題}

\subsection{ユークリッド空間上の滑らかな関数}
\begin{enumerate}
  \item \textgt{$C^2$級だが$C^3$級でない関数}\\
  $f\colon\mathbb{R}\to\mathbb{R}$を$f(x)=x^{1/3}$とし,
  $g\colon\mathbb{R}\to\mathbb{R}$を
  \begin{align}
    g(x)=\int_0^x f(t)dt=\int_0^xt^{1/3}dt=\frac{3}{4}x^{4/3}
  \end{align}
  とする.関数$h(x)=\int_0^xg(t)dt$は$C^2$級
  だが$x=0$で$C^3$級ではないことを示せ.
  \begin{proof}
    $h(x)$の1階微分,2階微分,3階微分を計算する.
    \begin{align}
      h'(x)&=g(x)=\frac{3}{4}x^{4/3},\\
      h''(x)&=g'(x)=f(x)=x^{1/3},\\
      h'''(x)&=f'(x)=\frac{1}{3}x^{-2/3}.
    \end{align}
    したがって,$h''(x)$は全ての$x$で定義されるが,
    $h'''(x)$は$x=0$で定義されない.
    よって,$h(x)$は$C^2$級だが$x=0$で$C^3$級ではない.
  \end{proof}
\end{enumerate}

\setcounter{subsection}{7}
\subsection{接空間}
\begin{enumerate}
  \setcounter{enumi}{9}
  \item \textgt{極大値}\\
  \label{secprob:8.10}
  多様体上の実数値関数$f\colon M\to\mathbb{R}$が
  $p\in M$において極大値をもつとは,
  $f(p)\geq f(q)$がすべての$q\in U$について成り立つような
  $p$の近傍$U$が存在することである.
  \begin{enumerate}
    \item 開区間$I$上で定義されている微分可能な関数
    $f\colon I\to\mathbb{R}$が$p\in I$において極大値をもつ
    ならば,$f'(p)=0$であることを示せ.
    \begin{proof}
      $f(p)\geq f(q)$がすべての$q\in I$について成り立つように
      $I$として取り直しても良い.\\
      $q<p$で$f(q)$が増加,$p<q$で$f(q)$が減少することを
      踏まえると以下2つが成り立つ.
      \begin{align}
        \lim_{q\to p^-}\frac{f(q)-f(p)}{q-p}&\leq 0\\
        \lim_{q\to p^+}\frac{f(q)-f(p)}{q-p}&\geq 0
      \end{align}
      $f$が微分可能であるため,左極限と右極限は
      等しくなければならず,$f'(p)=0$が従う.
    \end{proof}
    \item $C^\infty$級関数$f\colon M\to\mathbb{R}$が
    極大値をとる点は$f$の臨界点であることを証明せよ.
    \begin{proof}
      $p\in M$を$f$が極大値をとる点とし,
      $X_p\in T_p M$を接ベクトルとする.
      $c(t)$を始点$p$における速度ベクトルが$X_p$であるような
      $M$上の曲線とすると,
      $f\circ c\colon \mathbb{R}\to\mathbb{R}$
      は$0$において極大値をもち,(a)より
      \begin{align}
        &(f\circ c)'(0)=0\\
        \implies&
        (f\circ c)_*\qty(\frac{d}{dt})
        =f_{*,p}(X_p)=0
      \end{align}
      となる.これは任意の$X_p\in T_p M$について成り立つため,
      $f_{*,p}$は零写像であり,$p$は$f$の臨界点である.
    \end{proof}
  \end{enumerate}
\end{enumerate}


\end{document}