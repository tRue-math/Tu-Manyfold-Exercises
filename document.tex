\documentclass[dvipdfmx,a4paper,11pt]{jsarticle}

% 数式
\usepackage{amsmath,amsfonts,amssymb,mathtools,bm,titlesec,amsthm,ulem,siunitx,braket,url,physics,otf,bussproofs,enumitem,stmaryrd}
% 画像
\usepackage[dvipdfmx]{graphicx}
\usepackage{tikz}

\mathtoolsset{showonlyrefs} % 参照した式のみ番号を表示
\everymath{\displaystyle}

\renewcommand{\thesubsection}{\arabic{subsection}}
\usepackage{titlesec}
\titleformat{\subsection}[hang]{\normalfont\normalsize\bfseries}{\S\thesubsection}{1em}{}
\renewcommand{\theenumi}{\arabic{subsection}.\arabic{enumi}}
\renewcommand{\labelenumi}{\textbf{\theenumi.}}


\begin{document}

\title{Tu多様体 回答}
\author{tRue}
\date{\today}
\maketitle

\section*{節末問題}

\subsection{ユークリッド空間上の滑らかな関数}
\begin{enumerate}
  \item \textgt{$C^2$級だが$C^3$級でない関数}\\
  $f\colon\mathbb{R}\to\mathbb{R}$を$f(x)=x^{1/3}$とし,
  $g\colon\mathbb{R}\to\mathbb{R}$を
  \begin{align}
    g(x)=\int_0^x f(t)dt=\int_0^xt^{1/3}dt=\frac{3}{4}x^{4/3}
  \end{align}
  とする.関数$h(x)=\int_0^xg(t)dt$は$C^2$級
  だが$x=0$で$C^3$級ではないことを示せ.
  \begin{proof}
    $h(x)$の1階微分,2階微分,3階微分を計算する.
    \begin{align}
      h'(x)&=g(x)=\frac{3}{4}x^{4/3},\\
      h''(x)&=g'(x)=f(x)=x^{1/3},\\
      h'''(x)&=f'(x)=\frac{1}{3}x^{-2/3}.
    \end{align}
    したがって,$h''(x)$は全ての$x$で定義されるが,
    $h'''(x)$は$x=0$で定義されない.
    よって,$h(x)$は$C^2$級だが$x=0$で$C^3$級ではない.
  \end{proof}
\end{enumerate}


\end{document}